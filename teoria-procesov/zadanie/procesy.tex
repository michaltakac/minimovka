% !TeX root=teoriaprocesovtakac.tex
% !TeX encoding = UTF-8
% !TeX spellcheck = sk_SK
\section{Procesy v kyslíkovom konvertore}

Kroky, ktoré sú súčasťou LD procesu:

- Vsádzka (charging)

- Fúkanie (blowing)

- Vzorkovanie (sampling)

- Tapping

- Slag off


Cílem kyslíkové výroby oceli je spálení (tj. oxidace) nežádoucích nečistot obsažených v kovové vsázce. Hlavními prvky, které tudíž přecházejí na oxidy jsou uhlík, křemík, mangan, fosfor a síra.

Účelem tohoto oxidačního procesu tedy je :

snížit obsah uhlíku na předepsanou úroveň ( z přibližně 4\% na méně než 1 \%, ale často níže)
upravit obsah potřebných cizích prvků
odstranit nežádoucí nečistoty v maximálně možné míře
Výroba oceli kyslíkovým pochodem je diskontinuální proces, který zahrnuje následující kroky :

přepravu a skladování taveniny horkého kovu
předúpravu taveniny horkého kovu (odsiřování)
oxidaci v kyslíkovém konvertoru (oduhličení a oxidaci nečistot)
úpravu sekundární metalurgií
odlévání (kontinuální a/nebo do ingotů)


Podstatou výroby ocele v kyslíkovom konvertore je oxidácia prvkov z kovonosnej vsádzky s kyslíkom fúkaným do konvertora. Oxidy týchto prvkov prechádzajú do trosky alebo odchádzajú vo forme konvertorového plynu (Obr. 30). Intenzita oxidácie jednotlivých prvkov závisí od ich chemickej afinity ku kyslíku.
Oxidácia uhlíka je jedným z najdôležitejších procesov. Uhlík sa v kove počas
oceliarenského pochodu oxiduje vplyvom kyslíka najmä na \ce{CO} a čiastočne na \ce{CO2} podľa reakcií

\begin{equation}
\ce{C + 1/2O2 -> CO}
\end{equation}
\begin{equation}
\ce{C + O2 -> CO2}
\end{equation}

Mangán sa v konvertore oxiduje na \ce{MnO}

\begin{equation}
\ce{Mn + 1/2O2 -> MnO}
\end{equation}

Fosfor je v oceli nežiaduci a oxiduje sa na \ce{P2O5}

\begin{equation}
\ce{2P + 5/2O2 -> P2O5}
\end{equation}

Síra patrí medzi škodlivé prvky a prechádza do trosky vo forme \ce{CaS} na základe reakcie \ce{CaO}

\begin{equation}
\ce{CaO + MnS -> CaS + MnO}
\end{equation}

pričom \ce{MnS} vzniká podľa reakcie

\begin{equation}
\ce{Mn + S -> MnS}
\end{equation}

a síra taktiež odchádza aj vo forme plynu ako \ce{SO2}

\begin{equation}
\ce{S + O2 -> SO2}
\end{equation}

Kremík ma vysokú afinitu ku kyslíku, čiže sa ľahko oxiduje pričom vzniká \ce{SiO2}

\begin{equation}
\ce{Si + O2 -> SiO2}
\end{equation}

Potrebné je taktiež uvažovať aj straty železa vo forme \ce{FeO} a \ce{Fe2O3}

\begin{equation}
\ce{Fe + 1/2O2 -> FeO}
\end{equation}

\begin{equation}
\ce{2Fe + 3/2O2 -> Fe2O3}
\end{equation}

ktoré prechádzajú do trosky, resp. \ce{Fe2O3} odchádza v konvertorovom prachu. Kvapky kovového železa sa nachádzajú aj v troske (Obr. 30).


Vzniknutý SiO2 (29) prechádza do trosky ako \ce{2CaO.SiO2} podľa rovnice

\begin{equation}
\ce{SiO2 + 2CaO -> 2CaO.SiO2}
\end{equation}

a obdobne P2O5 (4) prechádza do trosky ako 3CaO.P2O5 podľa rovnice

\begin{equation}
\ce{P2O5 + 3CaO = 3CaO.P2O5}
\end{equation}

