% !TeX root=teoriaprocesovtakac.tex
% !TeX encoding = UTF-8
% !TeX spellcheck = sk_SK
\section{Procesy v kyslíkovom konvertore}

Podstatou výroby ocele v kyslíkovom konvertore je oxidácia prvkov z kovonosnej vsádzky s kyslíkom fúkaným do konvertora. Oxidy týchto prvkov prechádzajú do trosky alebo odchádzajú vo forme konvertorového plynu. LD proces sa skladá z~nasledujúcich elementárnych procesov:

\begin{enumerate}
	\item Vsádzanie šrotu
	\item Nalievanie tekutého surového železa
	\item Fúkanie kyslíka a pridávanie troskotvorných a legujúcich prísad
	\item Meranie teploty a zloženia ocele
	\item Odpich ocele
	\item Odpich trosky
\end{enumerate}

V moderných oceliarňach sa vyrobí cca 300t ocele v priebehu 30-40 minútového cyklu. Pre prispôsobenie akosti ocele a tvorbu trosky sa počas pochodu pridávajú rozličné prísady. Počas vsádzania a odpichu je konvertorová pec naklonená. Počas fúkania kyslíka má konvertor zvislú polohu. Zmeny polohy konvertora počas jednotlivých elementárnych procesov sú znázornené na obrázku \ref{o:30}.

\begin{figure}[h!]
	\centering
	\includegraphics[width=.35\textwidth,angle=0]{convertor-phases.jpg}
	\caption{Znázornenie elementárnych procesov v LD konvertore.}
	\label{o:30}
\end{figure}

V závislosti od miestnych prevádzkových podmienok, dostupnosti šrotu, vysokopecného železa a rozsahu predúpravy, je kovová vsádzka do konvertora (LD/BOF, Q-BOP) tvorená 75 až 95 \% surovým železom a zvyšok je oceľový šrot. Používané druhy šrotu sú zvyčajne tie, ktoré sa vyrábajú v oceliarni: šrot z plechu, poškodené formy, plechovky a podobne \cite{Turkdogan1996}.

Kyslík je fúkaný vysokou rýchlosťou (až do dvojnásobnej rýchlosti zvuku) na povrch kovového kúpeľa v konvertore a v oblasti povrchu sa vytvára tzv. horúce miesto, kde prúd kyslíka naráža na povrch. Oxidačné produkty sa rozpustia v troske s výnimkou oxidu uhoľnatého, ktorý prechádza vrstvou trosky a tvorí hlavnú zložku konvertovaného plynu. Intenzita oxidácie jednotlivých prvkov závisí od ich chemickej afinity ku kyslíku. Oxidácia uhlíka je jedným z najdôležitejších procesov. Uhlík sa v kove počas oceliarenského pochodu oxiduje vplyvom kyslíka najmä na \ce{CO} a~čiastočne na \ce{CO2} podľa reakcií

\begin{equation}
\ce{C + 1/2O2 -> CO}
\end{equation}
\begin{equation}
\ce{C + O2 -> CO2}
\end{equation}

Mangán sa v konvertore oxiduje na \ce{MnO}

\begin{equation}
\ce{Mn + 1/2O2 -> MnO}
\end{equation}

Fosfor je v oceli nežiaduci a oxiduje sa na \ce{P2O5}

\begin{equation}
\ce{2P + 5/2O2 -> P2O5}
\end{equation}

Síra patrí medzi škodlivé prvky a prechádza do trosky vo forme \ce{CaS} na základe reakcie \ce{CaO}

\begin{equation}
\ce{CaO + MnS -> CaS + MnO}
\end{equation}

pričom \ce{MnS} vzniká podľa reakcie

\begin{equation}
\ce{Mn + S -> MnS}
\end{equation}

a síra taktiež odchádza aj vo forme plynu ako \ce{SO2}

\begin{equation}
\ce{S + O2 -> SO2}
\end{equation}

Kremík ma vysokú afinitu ku kyslíku, čiže sa ľahko oxiduje pričom vzniká \ce{SiO2}

\begin{equation}
\ce{Si + O2 -> SiO2}
\end{equation}

V počiatočných fázach fúkania sa väčšina kremíka oxiduje za vzniku trosky nízkej zásaditosti - dochádza k zmene zloženia kovu a trosky, na čo poukazuje obrázok \ref{o:20}.

\begin{figure}[h!]
	\centering
	\includegraphics[width=.51\textwidth,angle=0]{slag-formation-data.jpg}
	\caption{Zmeny v zložení kovov a trosky počas výroby ocele v LD/BOF procesom pri 300t taveniny \citep{Turkdogan1996}.}
	\label{o:20}
\end{figure}

Intenzívny prúd kyslíka indukuje toky tekutín (cirkuláciu) v železnom kúpeli, následne núti vysoko oxidovaný kov a~roztavené oxidačné produkty z povrchu železného “kúpeľa” prenikať do vnútra kúpeľa, kde reagujú s~“čerstvým” kovom s~vysokým obsahom nečistôt a~preto je potrebné taktiež uvažovať aj~straty železa vo forme \ce{FeO} a~\ce{Fe2O3}

\begin{equation}
\ce{Fe + 1/2O2 -> FeO}
\end{equation}

\begin{equation}
\ce{2Fe + 3/2O2 -> Fe2O3}
\end{equation}.

Tento prúd kyslíka a plynové bubliny vznikajúce v kúpeli privádzajú časti železnej taveniny do trosky. Teplo vyvíjané pri vysoko exotermálnych oxidačných reakciách sa spotrebúva pri zahrievaní a tavení vsádzkových materiálov, zahrievaní železného kúpeľa, trosky a oxidov uhlíka, ktoré sa tvoria pri oxidácii uhlíka a čiastočne sa strácajú do okolia počas procesu fúkania. 

Vzniknutý \ce{SiO2} prechádza do trosky ako \ce{2CaO.SiO2} podľa rovnice

\begin{equation}
\ce{SiO2 + 2CaO -> 2CaO.SiO2}
\end{equation}

a obdobne \ce{P2O5} prechádza do trosky ako \ce{3CaO.P2O5} podľa rovnice \cite{sprava2017}

\begin{equation}
\ce{P2O5 + 3CaO = 3CaO.P2O5}.
\end{equation}

Cirkulácie v železnom kúpeli spôsobené prúdom kyslíka, stúpajúcimi bublinami plynu a preplachovaním inertného plynu cez spodné trubice v konvertoroch s kombinovaným typom fúkania sa transportujú minoritné zložky taveniny železa (C, Si, Ti, Mn, P, V atď.) do horných vrstiev kúpeľa \cite{Jalkanen2006}. 

\begin{figure}[h!]
	\centering
	\includegraphics[width=.9\textwidth,angle=0]{ld-convertor-processes-graphical.jpg}
	\caption{Chemické a tepelné procesy v LD konvertore \citep{Jalkanen2006}.}
	\label{o:25}
\end{figure}

\section{Riadenie procesov v kyslíkovom konvertore}

Keďže cieľom výroby ocele v kyslíkových konvertoroch je spálenie (tzv. oxidácia) nežiaducích nečistôt obsiahnutých v kovovej vsádzke, účelom tohto oxidačného procesu teda je:

\begin{itemize}
	\item znížiť obsah uhlíka na predpísanú úroveň (z približne 4 \% na menej ako 1 \%, ale často nižšie),
	\item upraviť obsah potrebných cudzích prvkov,
	\item odstrániť nežiadúce nečistoty v maximálne možnej miere.
\end{itemize}

Následnou úlohou riadiaceho procesu je potom získanie predpísaných parametrov ocele, ktorá sa odpichuje z konvertora, vrátane hmotnosti, teploty a obsahu každého prvku. Na základe týchto parametrov sa rozhoduje o tom, či je roztavená oceľ prijateľná alebo nie.

Počítačom podporované výpočty vsádzky sa robia pre každú tavbu. Asi 80 percent modelu riadenia vsádzky je založený na rovnováhe tepla a materiálu, zvyšok je založený na empirických vzťahoch, ktoré sa medzi jednotlivými taviarňami líšia. Pretože každá oceliareň má svoju vlastnú formuláciu modelu riadenia vsádzky \cite{Turkdogan1996}.

Za účelom monitorovania a riadenia procesu je možné použiť rôzne meracie systémy na poskytnutie spätnej väzby operátorovi alebo priamo existujúcemu systému na automatizované riadenie. Tieto merania môžu byť priame alebo nepriame, ako aj~s~časovým oneskorením alebo bez neho \cite{Widlund1998}.

\begin{figure}[h!]
\centering
\includegraphics[width=.9\textwidth,angle=0]{schematic-bof.jpg}
\caption{Schéma funkcionality systému automatizovaného LD procesu \citep{Turkdogan1996}.}
\label{o:21}
\end{figure}

Existuje len niekoľko procesných premenných, ktoré môže nastavovať riadiaci systém alebo obsluha - výška trysky pre prívod fúkaného kyslíka, prietok kyslíka a~prietok čistiaceho plynu. Zmeny výšky prívodnej trysky sa merajú a nastavujú ľahšie, a preto je lepšie ich používať v riadiacom systéme s uzavretou slučkou. Zmena prietoku kyslíka počas LD procesu nesmie byť väčšia ako 5\%, pretože dýza je navrhnutá pre špecifický prietok \cite{Widlund1998}.