\section*{Pr\'iloha A}
\addcontentsline{toc}{section}{\numberline{}Pr\'iloha A}
\subsection*{Pr\'ilohy}

Táto časť\/ záverečnej práce je povinná a~obsahuje zoznam všetkých
príloh vrátane elektronických nosičov. Názvy príloh v~zozname musia
byť\/ zhodné s~názvami uvedenými na príslušných prílohách. Tlačené
prílohy majú na prvej strane identifikačné údaje -- informácie zhodné
s~titulnou stranou záverečnej práce doplnené o~názov príslušnej
prílohy. Identifikačné údaje sú aj na priložených diskoch alebo
disketách. Ak je médií viac, sú označené aj číselne v~tvare $I/N$, kde
$I$ je poradové číslo a~$N$ je celkový počet daných médií. Zoznam
príloh má nasledujúci tvar:
\begin{description}
\item[Príloha A] CD médium -- záverečná práca v~elektronickej podobe,
prílohy v~elektronickej podobe.
\item[Príloha B] Používateľská príručka
\item[Príloha C] Systémová príručka
\end{description}
Prílohová časť\/ je samostatnou časťou kvalifikačnej práce. Každá
príloha začína na novej strane a je označená samostatným písmenom
(Príloha A, Príloha B, \dots). Číslovanie strán príloh nadväzuje na
číslovanie strán v~hlavnom texte. Pri každej prílohe sa má uviesť\/
prameň, z~ktorého sme príslušný materiál získali.