\documentclass[]{tukediphc}
%% -----------------------------------------------------------------
%% tento subor ma kodovanie utf-8
%%
%% na kompilaciu pouzivajte format pdfcslatex 
%%
%% vytvorene distribuciou texlive 2009-7, OS GNU/Linux
%% vytvorene distribuciou TeXLive 2010, OS Win XP
%% februar 2013
%% -----------------------------------------------------------------
\usepackage[utf8]{inputenc}
%\usepackage[T1]{fontenc}
\usepackage{lmodern,textcase}
\usepackage[slovak]{babel}\renewcommand{\figurename}{Obr\'azok}
\def\refname{Zoznam pou\v{z}itej literat\'ury}
\usepackage{latexsym}
\usepackage{dcolumn} % zarovnanie cisiel v tabulke podla des. ciarky
\usepackage{hhline}
\usepackage{amsmath}
\usepackage{nicefrac} % pekne zlomky
\usepackage{upgreek} % napr. $\upmu\mathrm{m}$ pre mikrometer ...
\usepackage[final]{showkeys}%color%notref%notcite%final
\usepackage[slovak,noprefix]{nomencl}
\makeglossary % prikaz na vytvorenie suboru .glo
\usepackage{parskip}% 'zhusti' polozky obsahu
%%
%\usepackage[dvips]{graphicx}
%\DeclareGraphicsExtensions{.eps}
\usepackage[pdftex]{graphicx}
\DeclareGraphicsExtensions{.pdf,.png,.jpg,.mps}
\graphicspath{{figures/}} % priecinok na obrazky
%%
%% Cislovane citovanie
%\usepackage[numbers]{natbib}
%%
%% Citovanie podľa mena autora a roku
\usepackage{natbib} %\citestyle{chicago}
% -----------------------------------------------------------------
%% tlač !!!
\usepackage[pdftex,unicode=true,bookmarksnumbered=true,
bookmarksopen=true,pdfmenubar=true,pdfview=Fit,linktocpage=true,
pageanchor=true,bookmarkstype=toc,pdfpagemode=UseOutlines,
pdfstartpage=1]{hyperref}
\hypersetup{%
baseurl={http://www.tuke.sk/sevcovic},
pdfcreator={pdfcsLaTeX},
pdfkeywords={Riadenie procesov, Oceliarstvo, Vizualizácia, Virtuálna realita, Matematické modelovanie},
pdftitle={Elaborát z predmetu Riadenie procesov},
pdfauthor={Michal Takáč},
pdfsubject={Dizertačná skúška}
} 
%% nehodiace zakomentujte !
%\dippraca{Elaborát z predmetu Riadenie procesov}
%\bakpraca{Príprava na dizertačnú skúšku}
%%
\nazov{Elaborát z predmetu Riadenie procesov}
%% ked praca nema 'podnazov' zakomentujte nasledujuci riadok
%% alebo polozku nechajte prazdnu
\podnazov{}
\autor{Ing.~Michal Takáč}
\veduciprace{prof.~Ing.~Ivo~Petráš, DrSc.}
\univerzita{Technická univerzita v~Košiciach}
\fakulta{Fakulta baníctva, ekológie, riadenia a geotechnológií}
\skratkafakulty{FBERG}
\katedra{Ústav riadenia a informatizácie výrobných procesov}
\skratkakatedry{URIVP}
\odbor{Riadenie procesov}
\specializacia{Kybernetika}
\abstrakt{Abstrakt je povinnou súčasťou každej práce. Je výstižnou
charakteristikou obsahu dokumentu. Nevyjadruje hodnotiace stanovisko
autora. Má byť\/ taký informatívny, ako to povoľuje podstata práce.
Text abstraktu sa píše ako jeden odstavec. Abstrakt neobsahuje odkazy
na samotný text práce. Mal by mať\/ rozsah 250 až 500 slov. Pri
štylizácii sa používajú celé vety, slovesá v činnom rode a tretej
osobe. Používa sa odborná terminológia, menej zvyčajné termíny,
skratky a~symboly sa pri prvom výskyte v texte definujú.}
\klucoveslova{Riadenie procesov, Oceliarstvo, Vizualizácia, Virtuálna realita, Matematické modelovanie}
\datumodovzdania{12. 5. 2020}
\mesto{Košice}

\begin{document}
\renewcommand\theHfigure{\theHsection.\arabic{figure}}
\renewcommand\theHtable{\theHsection.\arabic{table}}
\bibliographystyle{dcu}

\prvastrana


\thispagestyle{empty}
\tableofcontents
\newpage
%
%\thispagestyle{empty}
%%\addcontentsline{toc}{section}{\numberline{}Zoznam obrázkov}
%\listoffigures
%\newpage
%
%\thispagestyle{empty}
%%\addcontentsline{toc}{section}{\numberline{}Zoznam tabuliek}
%\listoftables
%\newpage

%%%%%%%%%%%%%%%%%%%%%%%%%%%%%%%%%

\setcounter{page}{1}
\setcounter{equation}{0}
\setcounter{figure}{0}
\setcounter{table}{0}

\section{Úvod}

The objective of process control is to keep key process-operating parameters within narrow bounds of the reference value or setpoint. Controllers are used to automate a human function in an effort to control a variable. A basic controller can keep an individual loop on an even point, so long as there is not too much disruption. Complex processes like ones in metallurgy might employ dozens or even hundreds of such controllers, but keeping an~eye on the big picture was, until not so long ago, a human process \cite{Al-Megren2016}.

Cieľom riadenia procesu je udržiavať kľúčové parametre prevádzky procesu v úzkom rozmedzí referenčnej hodnoty alebo požadovanej hodnoty. Ovládače sa používajú na automatizáciu ľudskej funkcie v snahe ovládať premennú. Základný ovládač môže udržiavať jednotlivú slučku na rovnomernom mieste, pokiaľ nedôjde k prílišnému prerušeniu. Komplexné procesy, ako sú procesy v metalurgii, môžu zamestnávať desiatky alebo dokonca stovky takýchto regulátorov, ale pozor na celkový obraz bol až donedávna ľudským procesom \cite{Al-Megren2016}

Although a device was used to automate a human function in an effort to control a variable, there was no sense of what the process was doing overall. A basic controller could keep an individual loop on an even keel, more or less, so long as there was not too much disruption. Complex processes might employ dozens or even hundreds of such controllers, each with its performance displayed on a panel board, but keeping an eye on the big picture was still a human process.	

Aj keď sa zariadenie používalo na automatizáciu ľudskej funkcie v snahe ovládať premennú, nemal zmysel, čo tento proces celkovo robí. Základný ovládač by mohol udržiavať individuálnu slučku na rovnomernom kýli viac alebo menej, pokiaľ nenastane príliš veľa prerušenia. Komplexné procesy môžu využívať desiatky alebo dokonca stovky takýchto kontrolérov, z ktorých každý má svoj výkon zobrazený na doske, ale pozor na celkový obraz bol stále ľudský proces.

The need for developing improved control systems has traditionally been powered by the demand for more accurate and cost efficient production. This is still a major driving force but environmental issues do also have a profound influence on this development today (\cite{Widlund1998}).

Potreba vývoja zdokonalených systémov riadenia bola tradične poháňaná požiadavkou presnejšej a nákladovo efektívnejšej výroby. Je to stále hlavná hnacia sila, ale environmentálne otázky majú na tento vývoj zásadný vplyv aj dnes (\cite{Widlund1998}).

The main objective of controlling oxygen converter steelmaking is to obtain prescribed parameters for the steel when it is tapped from the furnace, including weight, temperature, and each element content. In practical steelmaking process, the criterion whether the molten steel is acceptable or not is often decided by the endpoint carbon content and temperature (\cite{Wang2010}).

Hlavným cieľom riadenia výroby ocele s kyslíkovým konvertorom je získanie predpísaných parametrov pre oceľ, keď sa odoberá z pece, vrátane hmotnosti, teploty a obsahu každého prvku. V praktickom procese výroby ocele sa o konečnom obsahu uhlíka a teplote často rozhoduje o tom, či je roztavená oceľ prijateľná alebo nie \cite{Wang2010}.

Generally, the LD/BOF steelmaking process with sub-lance system can be divided into two stages: static control and dynamic control. Static models include oxygen supplying model, slaging model and bottom blowing model; dynamic models include decarburization speed model, molten steel warming model and the model for the amount of coolant. (\cite{Wang2010}).

Všeobecne možno povedať, že proces výroby ocele LD / BOF s pomocným systémom sa dá rozdeliť do dvoch stupňov: statické riadenie a dynamické riadenie. Statické modely zahŕňajú model prívodu kyslíka, model trosky a model vyfukovania dna; Medzi dynamické modely patrí model rýchlosti oduhličovania, model otepľovania roztavenej ocele a model množstva chladiva \cite{Wang2010}.

The fast dynamics of the LD converter steelmaking process or the BOF process, as it is commonly known, often makes it a challenge to obtain stable blowing conditions and to achieve the required steel composition and temperature simultaneously at the end point. For this reason, process control becomes very necessary and attempts had started as early as in the 1970s (\cite{Fritz2005}). Out of the originally very simple LD process have grown the modern process-controlled and automated production systems that enable present-day adaptations to meet today’s economic and ecological demands (\cite{Sarkar2015}). The non-linear nature of chemical and thermodynamical processes in basic oxygen steelmaking also amassed interest in developing new mathematical models based on fractionalorder calculus.

Rýchla dynamika procesu výroby ocele konvertorom LD alebo procesu BOF, ako je všeobecne známe, často spôsobuje, že je potrebné dosiahnuť stabilné podmienky vyfukovania a súčasne dosiahnuť požadované zloženie ocele a teplotu v koncovom bode. Z tohto dôvodu je riadenie procesu veľmi potrebné a pokusy sa začali už v sedemdesiatych rokoch (\cite{Fritz2005}). Z pôvodne veľmi jednoduchého procesu LD vznikli moderné a automatizované výrobné systémy riadené procesmi, ktoré umožňujú súčasné prispôsobenie sa dnešným hospodárskym a ekologickým požiadavkám (\cite{Sarkar2015}). Nelineárna povaha chemických a termodynamických procesov pri výrobe kyslíka v základnom meradle tiež vzbudila záujem o vývoj nových matematických modelov založených na zlomku počtu.

\url{https://www.primetals.com/fileadmin/user_upload/content/01_portfolio/2_steelmaking/converter-carbon-steelmaking/Converter_steelmaking_automation.pdf}



\section{Procesy v kyslíkovom konvertore}

Podstatou výroby ocele v kyslíkovom konvertore je oxidácia prvkov z kovonosnej vsádzky s kyslíkom fúkaným do konvertora. Oxidy týchto prvkov prechádzajú do trosky alebo odchádzajú vo forme konvertorového plynu. LD proces sa skladá z~nasledujúcich elementárnych procesov:

\begin{enumerate}
	\item Vsádzanie šrotu
	\item Nalievanie tekutého surového železa
	\item Fúkanie kyslíka a pridávanie troskotvorných a legujúcich prísad
	\item Meranie teploty a zloženia ocele
	\item Odpich ocele
	\item Odpich trosky
\end{enumerate}

V moderných oceliarňach sa vyrobí cca 300t ocele v priebehu 30-40 minútového cyklu. Pre prispôsobenie akosti ocele a tvorbu trosky sa počas pochodu pridávajú rozličné prísady. Počas vsádzania a odpichu je konvertorová pec naklonená. Počas fúkania kyslíka má konvertor zvislú polohu. Zmeny polohy konvertora počas jednotlivých elementárnych procesov sú znázornené na obrázku \ref{o:30}.

\begin{figure}[h!]
	\centering
	\includegraphics[width=.35\textwidth,angle=0]{convertor-phases.jpg}
	\caption{Znázornenie elementárnych procesov v LD konvertore.}
	\label{o:30}
\end{figure}

V závislosti od miestnych prevádzkových podmienok, dostupnosti šrotu, vysokopecného železa a rozsahu predúpravy, je kovová vsádzka do konvertora (LD/BOF, Q-BOP) tvorená 75 až 95 \% surovým železom a zvyšok je oceľový šrot. Používané druhy šrotu sú zvyčajne tie, ktoré sa vyrábajú v oceliarni: šrot z plechu, poškodené formy, plechovky a podobne \cite{Turkdogan1996}.

Kyslík je fúkaný vysokou rýchlosťou (až do dvojnásobnej rýchlosti zvuku) na povrch kovového kúpeľa v konvertore a v oblasti povrchu sa vytvára tzv. horúce miesto, kde prúd kyslíka naráža na povrch. Oxidačné produkty sa rozpustia v troske s výnimkou oxidu uhoľnatého, ktorý prechádza vrstvou trosky a tvorí hlavnú zložku konvertovaného plynu. Intenzita oxidácie jednotlivých prvkov závisí od ich chemickej afinity ku kyslíku. Oxidácia uhlíka je jedným z najdôležitejších procesov.

\section{Riadenie procesov v kyslíkovom konvertore}

Keďže cieľom výroby ocele v kyslíkových konvertoroch je spálenie (tzv. oxidácia) nežiaducích nečistôt obsiahnutých v kovovej vsádzke, účelom tohto oxidačného procesu teda je:

\begin{itemize}
	\item znížiť obsah uhlíka na predpísanú úroveň (z približne 4 \% na menej ako 1 \%, ale často nižšie),
	\item upraviť obsah potrebných cudzích prvkov,
	\item odstrániť nežiadúce nečistoty v maximálne možnej miere.
\end{itemize}

Následnou úlohou riadiaceho procesu je potom získanie predpísaných parametrov ocele, ktorá sa odpichuje z konvertora, vrátane hmotnosti, teploty a obsahu každého prvku. Na základe týchto parametrov sa rozhoduje o tom, či je roztavená oceľ prijateľná alebo nie.

Počítačom podporované výpočty vsádzky sa robia pre každú tavbu. Asi 80 percent modelu riadenia vsádzky je založený na rovnováhe tepla a materiálu, zvyšok je založený na empirických vzťahoch, ktoré sa medzi jednotlivými taviarňami líšia. Pretože každá oceliareň má svoju vlastnú formuláciu modelu riadenia vsádzky \cite{Turkdogan1996}.

Za účelom monitorovania a riadenia procesu je možné použiť rôzne meracie systémy na poskytnutie spätnej väzby operátorovi alebo priamo existujúcemu systému na automatizované riadenie. Tieto merania môžu byť priame alebo nepriame, ako aj~s~časovým oneskorením alebo bez neho \cite{Widlund1998}.

\begin{figure}[h!]
	\centering
	\includegraphics[width=.9\textwidth,angle=0]{figures/schematic-bof.jpg}
	\caption{Schéma funkcionality systému automatizovaného LD procesu \citep{Turkdogan1996}.}
	\label{o:21}
\end{figure}

Existuje len niekoľko procesných premenných, ktoré môže nastavovať riadiaci systém alebo obsluha - výška trysky pre prívod fúkaného kyslíka, prietok kyslíka a~prietok čistiaceho plynu. Zmeny výšky prívodnej trysky sa merajú a nastavujú ľahšie, a preto je lepšie ich používať v riadiacom systéme s uzavretou slučkou. Zmena prietoku kyslíka počas LD procesu nesmie byť väčšia ako 5\%, pretože dýza je navrhnutá pre špecifický prietok \cite{Widlund1998}.

\section{Automatizované riadenie}

Plne optimalizovaná automatizácia elektrických systémov je základom pre spoľahlivé výrobné procesy, maximálny výkon zariadenia a kvalitné výrobky, ktoré vyhovujú všetkým požiadavkám trhu.


Systémy optimalizácie procesov zahŕňajú pokročilé procesné modely, využitie umelej inteligencie, grafické užívateľské rozhrania pre manuálne riadenie procesov človekom (SCADA, HMI, vizualizácie dát) a prevádzkové odborné znalosti. Procesné modely optimalizujú rôzne výrobné procesy so zreteľom na zníženie spotreby energie a emisií.

Automatizácia

Základnú časť automatizácie LD procesu


Naklápací pohon kovertora

Úlohou naklápacích pohonov je rotácia nádoby konvertora do plniacej, vyprázdňovacej alebo vzorkovacej polohy.

Systém s uzavretou slučkou (closed-loop)



Miešanie zdola - ovládanie jedným vedením
Riadiaci systém kyslíkovej dýzy

\subsubsection{Systém na odoberanie vzoriek taveniny a meranie teploty}


Odberná sonda je dôležitým nástrojom pri výrobe ocele LD procesom. Poskytuje operátorovi cenné informácie o procese s prevodníkom stále vo zvislej polohe. Zvyčajne sa dve odčítania Sublance uskutočňujú počas úderu, jeden počas úderu pre údaje v procese a jeden na konci tepla na konečné odčítanie. Normálne sa meria teplota, uhlík a troska. Výška oceľového kúpeľa a úrovne trosky sú tiež realizované pomocou systému Sublance.


Horizontálny merací manipulátor
Automatický odpichový systém
Pneumatická zarážka trosky
Manipulácia s materiálom
Váženie a kontrola prísad a zliatin
Chladenie a čistenie odpadového plynu
Zhodnotenie a analýza plynu
Blokovací a poplachový systém



%
%%
\Urlmuskip=0mu plus 1mu\relax
\bibliographystyle{spbasic}
\bibliography{references/interaction.bib,references/process.bib,references/programming.bib,references/simulation.bib,references/modeling.bib,references/visualization.bib,references/vr.bib,references/uncategorized.bib}
%

\end{document}
%%